\subsection{EFT intereprations}

Effective Field Theory (EFT) is an model independent way to parametrize the high enregy scale effects in the enregy scale available to us. The general form of the lagrangian is:

\begin{equation}
    \mathcal{L} = \mathcal{L}_{SM} + \mathcal{L}^{5} + \mathcal{L}^{6} + \mathcal{L}^{7} + \ldots,\quad \mathcal{L}^{(d)} = \sum_{i=1}^{n_d} \frac{C_i^{(d)}}{\Lambda^{d-4}}Q_i^{(d)} \quad \textrm{for}~ d>4,
\end{equation}

Where $\Lambda$ is the New Physics (NP) enregy scale, the parameter $C_{i}{(d)}$ are the Wilson Coefficients.

One of most promissing EFT model is the SMEFT \cite{Brivio:2017btx,Aebischer:2017ugx}. Since the operators $Q_{i}^{(d)}$ are surpresssed by the power of cutoff scale $\Lambda$, so we will work with dimension-6 operators only.

Currently, we are trying to produce the Leading Order (LO) ggH process with additional jets upto 2 jets. Like:

\begin{verbatim}
import model SMEFTsim_A_general_MwScheme_UFO_v2
#import model SMEFTsim_A_general_alphaScheme_UFO_v2
generate p p > h QED=1 NP<=1 @0
add process p p > h j QED=1 NP<=1 @1
add process p p > h j j QED=1 NP<=1 @2
\end{verbatim}

Our plan with this is following:

\begin{itemize}
    \item Generate SM from the SMEFT model and compare it with the NNNLOPS official samples (from HIG-19-001).
    \item Decide the set of parameter for which our analysis is sensitive.
    \item Validate the reweight method for our model.
    \item After finalizing previous step we will try to submit for official full CMSSW simulation.
\end{itemize}