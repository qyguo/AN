Referred to \cite{CMS-PAS-HIG-19-001}.

%More details on electron reconstruction can be found in Ref.~\cite{ElectronLegacy}. 
%
%Electron candidates are preselected using loose cuts on track-cluster matching observables, so as to preserve the highest possible efficiency while rejecting part of the QCD background. To be considered for the analysis, electrons are required to have a
%transverse momentum $p^e_T >$ 7 GeV, a reconstructed $|\eta^e| <$ 2.5, and to satisfy a loose primary vertex 
%constraint defined as $d_{xy} < 0.5$ cm and $d_z < 1$ cm.
%Such electrons are called {\bf loose electrons}.
%
%The data-MC discrepancy is corrected using scale factors as is done for the electron selection with data efficiencies measured using the same tag-and-probe technique outlined later (see Section~\ref{sec:eleEffMeas}). 
%These studies for reconstructions are carried out by the EGM POG and the results are only summarised here.
%
%The electron reconstruction scale factors are shown Fig.~\ref{fig:ele_rec_scale_factors} and are applied as a function of the super cluster $\eta$ and electron $\pt$.
%
%\begin{figure}[!htb]
%\vspace*{0.3cm}
%\begin{center}
%\includegraphics[width=0.4\textwidth]{Figures/Electrons/ErecoPt}
%\includegraphics[page=2, width=0.4\textwidth]{Figures/Electrons/ErecoEta}\\
%\includegraphics[width=0.4\textwidth]{Figures/Electrons/ErecoPt_lowPt}
%%\includegraphics[width=0.4\textwidth]{Figures/Electrons/ErecoEta_lowPt}
%\end{center}
%\caption{Electron reconstruction efficiencies efficiency in data versus $p_T$ (left) and $\eta$ (right) for electrons with $\pt > 20 \GeV$ (top) and $\pt < 20 \GeV$ (bottom) with corresponding data/MC scale factors as provided by the EGM POG. Errors are statistical only. }
%\label{fig:ele_rec_scale_factors}
%\end{figure}
%

