Referred to \cite{CMS-PAS-HIG-19-001}.

%The Tag-and-Probe study was performed on the EGM dataset using the golden JSON of ~58.83 fb$^{-1}$. More details on the Tag-and-Probe method can be found in Ref.~\cite{AN-15-277}. 
%
%Tag electrons need to satisfy the following quality requirements:
%\begin{itemize}
%%\item trigger matched to HLT\_Ele27\_eta2p1\_WPTight\_Gsf\_v*
%\item~trigger matched to single electron trigger (e.g HLT\_Ele32\_WPTight\_Gsf\_L1DoubleEG\_v* for 2018 for instance)
%\item~$p_{T} > 30$~GeV (tag), super cluster (SC) $\eta < 2.17$% but on in EB-EE gap ($1.4442<|\eta|<1.566$)
%\item~the tag and the probe need to have opposite charge.
%%\item tight working point of the Spring16 cut-based electron ID
%\end{itemize}
%
%For the bin between 7 and 20 GeV, additional criteria are required: 
%\begin{itemize}
%\item~the tag has to pass a cut on the MVA ID$>0.92$, $\sqrt(2*PFMET*p_T(tag)*(1-cos(\phi_{MET}-\phi_{tag})))<45 GeV$.
%\item~tag $p_{T}$ increased to 50 GeV
%\item~the charge is determined with the so-called selection method, asing all three estimates of the electron charge to agree.
%\end{itemize}
%%t
%These additional requirements help cleaning the background and makes the fits more reliable (and thus, the measurement more precise).
%
%Probe electrons only need to be reconstructed as GsfElectron while the FSR recovery algorithm is not applied in efficiency measurement. 
%
%The nominal MC efficiencies are evaluated from the LO MadGraph Drell-Yan, while the NLO systematics use the MadGraph\_AMCatNLO sample (or POWHEG in 2018).
%
%For the efficiency measurements a template fit is used. The $m_{ee}$ signal shape of the passing and failing probes is taken from MC and convoluted with a Gaussian. The data is then fitted with the convoluted MC template and a CMSShape (an Error-function with a one-sided exponential tail). This change follows from the usage of the T\&P tool developed by the EGM POG. For the low $p_T$ bins, a gaussian is added to the signal model for the failing probes. 
%
%%\paragraph{Electron selection efficiency measurements}\mbox{}\\
%%\label{par:Efficiency_measurements}
%
%The electron selection efficiency is measured as a function of the probe electron $p_{T}$ and its SC $\eta$, and separately for electrons falling in the ECAL gaps. Figure \ref{fig:ele_sel_pt_turn_onA},~\ref{fig:ele_sel_pt_turn_onB},~\ref{fig:ele_sel_pt_turn_onC} and~\ref{fig:ele_sel_eta_turn_onA},\ref{fig:ele_sel_eta_turn_onB},~\ref{fig:ele_sel_eta_turn_onC} show the $p_{T}$ and $\eta$ turn-on curves measured in data, for 2016, 2017 and 2018.
%% and the final 2D scale factor is shown in Fig.~\ref{fig:ele_sel_scale_factors} together with the systematic uncertainties. %These scale factors are very similar to the ICHEP figures, but more homogenous across $\eta$ and $p_{T}$ because of the higher statistics and the usage of more stable fitting routines in the new T\&P tool.
%
%%\includegraphics[page=2, width=0.4\textwidth]{Figures/Electrons/ErecoEta}\\
%
%\begin{figure}[!htb]
%\begin{center}
%    \subfigure [] {\resizebox{7.5cm}{!}{\includegraphics[page=1]{Figures/Electrons/2016_egammaEffitxt_egammaPlots.pdf}}} %eleSFvspT.png}}}
%    \subfigure [] {\resizebox{7.5cm}{!}{\includegraphics[page=1]{Figures/Electrons/2016_egammaEffitxt_egammaPlotsGAP.pdf}}} \\
%    %    \subfigure [] {\resizebox{7.5cm}{!}{\includegraphics{Figures/Electrons/eleSFvspTgap.png}}}\\
%\caption{Electron selection efficiencies vs $p_T$ measured using the Tag-and-Probe technique described in the text, non-gap electrons (left) and gap electrons (right), together with the corresponding data/MC ratio (bottom), for 2016 samples.}
%\label{fig:ele_sel_pt_turn_onA}
%\end{center}
%\end{figure}
%
%\begin{figure}[!htb]
%\begin{center}
%    \subfigure [] {\resizebox{7.5cm}{!}{\includegraphics[page=2]{Figures/Electrons/2016_egammaEffitxt_egammaPlots.pdf}}}
%    \subfigure [] {\resizebox{7.5cm}{!}{\includegraphics[page=2]{Figures/Electrons/2016_egammaEffitxt_egammaPlotsGAP.pdf}}}\\
%\caption{Electron selection efficiencies vs $\eta$ measured using the Tag-and-Probe technique described in the text, non-gap electrons (left) and gap electrons (right), together wit the corresponding data/MC ratio at the bottom of each plot, for 2016 samples. Dashed lines is MC, solid lines is DATA.}
%\label{fig:ele_sel_eta_turn_onA}
%\end{center}
%\end{figure}
%
%\begin{figure}[!htb]
%\begin{center}
%    \subfigure [] {\resizebox{7.5cm}{!}{\includegraphics[page=1]{Figures/Electrons/2017_egammaEffitxt_egammaPlots.pdf}}} %eleSFvspT.png}}}
%    \subfigure [] {\resizebox{7.5cm}{!}{\includegraphics[page=1]{Figures/Electrons/2017_egammaEffitxt_egammaPlotsGAP.pdf}}} \\
%    %    \subfigure [] {\resizebox{7.5cm}{!}{\includegraphics{Figures/Electrons/eleSFvspTgap.png}}}\\
%\caption{Electron selection efficiencies vs $p_T$ measured using the Tag-and-Probe technique described in the text, non-gap electrons (left) and gap electrons (right), together with the corresponding data/MC ratio (bottom), for 2017 samples.}
%\label{fig:ele_sel_pt_turn_onB}
%\end{center}
%\end{figure}
%
%\begin{figure}[!htb]
%\begin{center}
%    \subfigure [] {\resizebox{7.5cm}{!}{\includegraphics[page=2]{Figures/Electrons/2017_egammaEffitxt_egammaPlots.pdf}}}
%    \subfigure [] {\resizebox{7.5cm}{!}{\includegraphics[page=2]{Figures/Electrons/2017_egammaEffitxt_egammaPlotsGAP.pdf}}}\\
%\caption{Electron selection efficiencies vs $\eta$ measured using the Tag-and-Probe technique described in the text, non-gap electrons (left) and gap electrons (right), together wit the corresponding data/MC ratio at the bottom of each plot, for 2017 samples. Dashed lines is MC, solid lines is DATA.}
%\label{fig:ele_sel_eta_turn_onB}
%\end{center}
%\end{figure}
%
%\begin{figure}[!htb]
%\begin{center}
%    \subfigure [] {\resizebox{7.5cm}{!}{\includegraphics[page=1]{Figures/Electrons/2018_egammaEffitxt_egammaPlots.pdf}}} %eleSFvspT.png}}}
%    \subfigure [] {\resizebox{7.5cm}{!}{\includegraphics[page=1]{Figures/Electrons/2018_egammaEffitxt_egammaPlotsGAP.pdf}}} \\
%    %    \subfigure [] {\resizebox{7.5cm}{!}{\includegraphics{Figures/Electrons/eleSFvspTgap.png}}}\\
%\caption{Electron selection efficiencies vs $p_T$ measured using the Tag-and-Probe technique described in the text, non-gap electrons (left) and gap electrons (right), together with the corresponding data/MC ratio (bottom), for 2018 samples.}
%\label{fig:ele_sel_pt_turn_onC}
%\end{center}
%\end{figure}
%
%\begin{figure}[!htb]
%\begin{center}
%    \subfigure [] {\resizebox{7.5cm}{!}{\includegraphics[page=2]{Figures/Electrons/2018_egammaEffitxt_egammaPlots.pdf}}}
%    \subfigure [] {\resizebox{7.5cm}{!}{\includegraphics[page=2]{Figures/Electrons/2018_egammaEffitxt_egammaPlotsGAP.pdf}}}\\
%\caption{Electron selection efficiencies vs $\eta$ measured using the Tag-and-Probe technique described in the text, non-gap electrons (left) and gap electrons (right), together wit the corresponding data/MC ratio at the bottom of each plot, for 2018 samples. Dashed lines is MC, solid lines is DATA.}
%\label{fig:ele_sel_eta_turn_onC}
%\end{center}
%\end{figure}
%
%%\begin{figure}[!htb]
%%\begin{center}
%%    \subfigure [] {\resizebox{7.5cm}{!}{\includegraphics{Figures/Placeholder.png}}}
%%    \subfigure [] {\resizebox{7.5cm}{!}{\includegraphics{Figures/Placeholder.png}}}\\
%%\caption{2D ($p_T, \eta$) Electron selection Scale Factors measured using the Tag-and-Probe technique described in the text, non-gap electrons (left) and gap electrons (right).}
%%\label{fig:ele_sel_scale_factors}
%%\end{center}
%%\end{figure}
%
%
%%\paragraph{Systematic uncertainties}\mbox{}\\
%%\label{par:Systematic_uncertainties}
%%%%%%%%%%%%%%%%%%%%%%%%%%%%%
%
% The EGM recommendations on the evaluation of Tag-and-Probe uncertainties for efficiency measurements are followed. Specifically, we consider
%
%\begin{itemize}
%   \item Variation of the signal shape from a MC shape to an analytic shape (Crystal Ball) fitted to the MC
%   \item Variation of the background shape from a CMS-shape to a simple exponential in fits to data
%%   \item Variation of the tag selection: tag $p_{T}>$35~GeV and passes MVA-based 8X ID
%   \item Using an NLO MC sample for the signal templates
%\end{itemize}
%
%The total uncertainty for the measurement of the scale factors is the quadratic sum of the statistical uncertainties returned from the fit and the aforementioned systematic uncertainties.
%
%\clearpage
%
%
%
