\section{Introduction}
\label{sec:intro}

The ATLAS and CMS collaborations first reported the discovery of a new boson in 2012~\cite{Chatrchyan:2012ufa,Aad:2012tfa}
consistent with the standard model (SM) Higgs 
boson~\cite{Englert:1964et,Higgs:1964ia,Higgs:1964pj,Guralnik:1964eu,Higgs:1966ev,Kibble:1967sv} 
based on proton-proton ($\Pp\Pp$) collisions delivered by the CERN LHC at a center-of-mass energy of 
$\sqrt{s}=7\TeV$ in 2011 and $8\TeV$ in 2012. 
Subsequent studies by CMS using the full LHC Run 1 data set in various decay channels and production modes 
and combined measurements from ATLAS and CMS ~\cite{CMS:2014ega, AtlasProperties, Aad:2015zhl, CMS:2015kwa}
showed that the properties of the new boson are so far consistent with expectations for the SM Higgs boson.

The $\HZZfl$ decay channel ($\ell=\Pe,\Pgm$) has a large signal-to-background ratio due to the 
complete reconstruction of the final state decay products and excellent lepton momentum resolution
and is one of the most important channels for studies of the Higgs boson's properties. 
Measurements performed using this decay channel and the Run 1 data set include the determination of 
the mass and spin-parity of the new boson~\cite{CMSH4lLegacy,CMSH4lSpinParity,CMSH4lAnomalousCouplings}, 
its width~\cite{CMSH4lWidth,CMSH4lLifetime} and fiducial cross sections~\cite{CMSH4lFiducial8TeV}, as well as 
tests for anomalous HVV couplings~\cite{CMSH4lAnomalousCouplings,CMSH4lLifetime}. 

This analysis note presents measurements of properties of the Higgs boson in the $\HZZfl$ decay channel at $\sqrt{s}=13\TeV$
using $\usedLumiABC$ of $\Pp\Pp$ collision data collected with the CMS experiment at the LHC in 2016, 2017 and 2018.
Compared to the previous public result~\cite{CMS-PAS-HIG-09-001}, the full available dataset has been fully re-analyzed, with several improvements: 
\begin{itemize}
\item~analysis of re-recoed 2016 data (\textit{17July2018}~\cite{PPDtwiki})  and v2 version of the Fall17MiniAOD for 2017 MC samples.
\item~estimation of rare backgrounds (tt+V, tt+VV, VVV,... where V stands for W or Z) from MC samples (see Section~\ref{sec:rare}).
\item~development of a BDT combining identification and isolation observables to improve electron selection performance (2016), similar to what was already done for 2017 and 2018 (see Section~\ref{sec:ele_MVA}).
%\item~developpment of a BDT combining identification and isolation observables to improve muon selection performance (2016, 2017, 2018),
\item~improved measurement of lepton scale factors (see Sections~\ref{sec:eleEffMeas},~\ref{sec:muonEffMeas}). % aiming at smaller uncertainty, especially for electrons at low momenta,
\item~in-depth studies of jets treatment (implementation of L1 pre-firing emulation for 2016, 2017, removal of noisy jets in 2017, impact assessment of HEM 16/17 failures in 2018, new and improved Jet Energy Scale and Resolutions corrections for 2018), see Section~\ref{sec:jetstudies}.
\item~Jet PU-ID properly applied (only for jets with $30<p_T<50$ GeV) as recommended by the JetMET POG.
\item~DeepCSV b-tagging algorithm used for all three years. 
%\item~measurement of STXS Stage 1.2 (instead of 1.1 before), see Section~\ref{subsec:stxs}.
\item~better binning for differential measurement (especially $\pt(\PH)$).
\item~several additional (production and as well as decay) observables studied with optimized binning.
\item~in addition to \sc{powheg} and \sc{NNLOPS}, \sc{MADGRAPH5} is used as additional theory prediction to for comparision of the observed results.
\item~Effective Field Theory (EFT) interpretation of differential measurements are also made.
\end{itemize}

%The data collected in 2018 corresponding to $\usedLumiC$ is analyzed carefully and combined with $\usedLumiB$ of the data from 
%2017~\cite{CMS-PAS-HIG-18-001} and with $\usedLumiA$ of data from 2016~\cite{CMS-HIG-16-041} that were previously analyzed. Therefore, the main
%focus of this analysis note is to present the results of the analysis of 2018 data that is later on combined on a datacard level
%with the previously analyzed 2016 and 2017 data.





