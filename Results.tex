\section{Results}

%\textbr{FIXME: Results need to be updated with latest updates. Still results from the PAS}
In this section, we describe the methods used for the estimation of the significance of the excess of events above the SM backgrounds observed in the low-mass region, and of the signal strength, i.e. its cross section normalized to the one expected for a SM Higgs.

To exploit all the properties of the resonance under study or search, a multi-dimensional fit is implemented.
%For each of the categories defined in Section~\ref{sec:categorization}, two variables are used in the maximum likelihood fit, namely:
For each of the categories defined in Section~\ref{sec:categorization}, the variable used in the maximum likelihood fit is the four-lepton mass without kinematic refitting $\mllll$. 
%\begin{enumerate}
%\item The four-lepton mass without kinematic refitting $\mllll$, %or the refitted mass $m_{4\ell}'$ ;
%%\item A kinematical discriminant $K_D$, where $K_D = \DbkgVBFdec$ for VBF-2 jets tagged category, $\DbkgVHdec$ for VH-hadronic tagged category and $\KD$ for all other categories.
%% $\KD$, except in VBF-2jet and VH-hadronic categories where $\DbkgVBFdec$ and $\DbkgVHdec$ are used instead. (\textbf{FIXME: not yet in the results below}).
%\end{enumerate}

%To account for the strong correlation of the kinematic discriminant with the mass, a 2D histogram template of $K_D$ vs. $\mllll$ is implemented.
%Due to the small number of expected events in the mass peak, the mass dimension is unbinned and the resolution model is used as described in section \ref{sec:signalshapes}. 
%The total PDF is defined as:
%\begin{equation}
%%\mathcal{L}_{2D}(m_{4l},\KD)  =\mathcal{L}(m_{4l}) \mathcal{L}(\KD|m_{4l}) ,
%\mathcal{L}_{2D}(m_{4l},K_D)  =\mathcal{L}(m_{4l}) \mathcal{L}(K_D|m_{4l}) ,
%\end{equation}
%where the first factor corresponds to the 1D mass PDF and the second factor to the 2D template of mass vs. kinematic discriminant. 
%The conditional term in the second factor is implemented in the template by normalizing all columns corresponding to the same mass to the same value each. 
%Therefore, the 2D template doesn't include any information on the mass, but given the mass, it provides information on the kinematic discriminant.
%2D Templates for $\DbkgVBFdec$ and $\DbkgVHdec$ vs $m_{4l}$ are computed in 7x20 bins, for all processes. 
%2D Templates for $\KD$ vs $m_{4l}$ are computed in 35x30 bins in the Untagged categories. The same templates are also used for VH-leptonic and ttH tagged categories.
%Based on the seven event categories and the three final states ($4\mu$, $4e$, $2e2\mu$), the $(\mllll, K_D)$ unbinned distributions of selected events are split into $22\times3=66$ categories.
%
%%\subsection{Signal Strength}
%%\label{sec:sign}
%\subsection{Signal strength}
\label{subsec:signalstrenghts}

A simultaneous fit to all categories is performed to extract the signal-strength modifier,
defined as the ratio of the observed H boson rate in the $\mathrm{H}\rightarrow{\rm Z}{\rm Z}\rightarrow4\ell$ decay channel to the standard model expectation.

All measurements are reported at $\mH=125.38$ GeV, the best mass obtained by CMS from the combination of $H\rightarrow$~ZZ and $H\rightarrow\gamma\gamma$ channels.

The combined measurement of the inclusive signal-strength modifier is: % with $\mH$ profiled in the fit is
$\mu = 0.94~^{+0.11}_{-0.12} = 0.94^{+0.07}_{-0.07}~({\rm stat})~^{+0.09}_{-0.08}~({\rm syst})$.
%\begin{align}
%    \label{eqn:muinclusive}
%    \mu &= 0.96~^{+0.11}_{-0.10} \\
%    \nonumber &= 0.96^{+0.07}_{-0.07}~({\rm stat})~^{+0.09}_{-0.07}~({\rm syst})~.
%\end{align}
The dominant experimental sources of systematic uncertainty are the uncertainties in the lepton identification efficiencies and luminosity measurement, while the dominant theoretical source is the uncertainty in the total gluon fusion cross section.
The contributions to the total uncertainty from experimental and theoretical sources are found to be similar in magnitude.
The signal-strength modifiers are further studied in terms of the five main SM Higgs boson production mechanisms, namely $\mu_{\Pg\Pg\PH,\bbH}$, $\mu_{\mathrm{VBF}}$, $\mu_{\ZH}$, $\mu_{\WH}$, and $\mu_{\ttH,\tH}$.
%The WH and ZH processes are merged into VH.
Contributions of the $\bbH$ and $\tH$ production modes are also taken into account in the fit.
The $\bbH$ contribution is floated together with gluon fusion and $\tH$ production mode is floated with $\ttH$.
The results are shown in Fig.~\ref{fig:mucat} reporting the observed and expected profile likelihood scans of the inclusive signal-strength modifier (left) and the results of likelihood scans for the signal-strength modifiers of the five main SM Higgs boson production mechanisms (right).
The corresponding numerical values, including the decomposition of the uncertainties into statistical and systematic components, and the corresponding expected uncertainties, are given in Table~\ref{tab:sigstr}.

%%%%%%%%%%%%%%%%%%%%%%%
\begin{figure}[!htb]
\begin{center}
\includegraphics[width=0.49\linewidth]{Figures/results/signalstrength/InclusiveMu_125_38.pdf}
\includegraphics[width=0.49\linewidth]{Figures/results/signalstrength/mu_stage0_125_38.pdf}
\caption{
(Left) The observed and expected profile likelihood scans of the inclusive signal-strength modifier. The scans are shown both with (solid line) and without (dashed line) systematic uncertainties.
(Right) Results of likelihood scans for the signal-strength modifiers corresponding to the five main SM Higgs boson production mechanisms, compared to the inclusive signal strength modifier  $\mu$ shown as a vertical line.
The thick black lines report the $1\sigma$ confidence intervals including both statistical and systematic sources.
The thick red lines report the statistical uncertainties of the $1\sigma$ confidence intervals.
\label{fig:mucat}}
\end{center}
\end{figure}
%%%%%%%%%%%%%%%%%%%%%%%

\begin{table}[hb]
	\begin{center}
		\caption{
		Best-fit values and $\pm 1\sigma$ uncertainties for the expected and observed signal-strength modifiers.
		The uncertainty numbers are broken into statistical and systematic sources.
		The expected results are obtained for $\mH=125.38~\mathrm{GeV}$. %the observed results are obtained with $\mH$ profiled in the fit.
		\label{tab:sigstr}
			}
    \renewcommand{\arraystretch}{1.5}
    \begin{tabular}{ccc}
	\hline
	& Expected & Observed \\
	\hline
	$\mu_{\Pg\Pg\PH,\bbH}$ & $1.00~^{+0.10}_{-0.10}$~(stat)$~^ {+0.12}_{-0.10}~$(syst) & $0.99 ^{+0.09 }_{-0.09}$(stat)~$~^ {+0.11}_{-0.09}~$(syst) \\
  $\mu_{\mathrm{VBF}}$ & $1.00~^{+0.53}_{-0.44}$~(stat)$~^ {+0.18}_{-0.12}~$(syst) & $0.48 ^{+0.46}_{-0.37}$(stat)~$~^ {+0.14}_{-0.10}~$(syst) \\
  $\mu_{\ZH}$ & $1.00~^{+4.79}_{-1.00}~$(stat)$~^ {+6.76}_{-0.00}~$(syst) & $0. ^{+4.38}_{-0.00}$(stat)~$~^ {+3.24}_{-0.00}~$(syst) \\
  $\mu_{\WH}$ & $1.00~^{+1.83}_{-1.00}~$(stat)$~^{+0.75}_{-0.00}~$(syst) & $1.66 ^{+1.52}_{-1.66}$(stat)~$~^ {+0.85}_{-0.00}~$(syst) \\
	$\mu_{\ttH,\tH}$ & $1.00~^{+1.23}_{-0.77}~$(stat)$~^ {+0.51}_{-0.06}~$(syst) & $0.17 ^{+0.88}_{-0.17}$(stat)~$~^{+0.42}_{-0.00}~$(syst) \\
		  	\hline
	$\mu$ & $1.00~^{+0.08}_{-0.07}$~(stat)~$^{+0.10}_{-0.08}$~(syst) & $0.94~^{+0.07}_{-0.07}$~(stat)~$^{+0.09}_{-0.08}$~(syst)\\
	\hline
\end{tabular}
	\end{center}
\end{table}

Two signal strength modifiers $\muF$ and $\muV$ are introduced as scale factors for the fermion- and vector-boson induced contribution to the expected SM cross section.
A two-parameter fit is performed simultaneously to all reconstructed event categories, leading to the measurements of $\muF=\valMuF$ and $\muV=\valMuV$. 
%profiling $\mH$, 
The expected meaurements obtained for $\mH=125.38~\mathrm{GeV}$ are $\muF~=~1.00~^{+0.15}_{-0.13}$ and $\muV~=~1.00~^{+0.39}_{-0.33}$.
The 68\% and 95\%~CL contours in the ($\muF,\muV$) plane are shown in Fig.~\ref{fig:mu2D} and the SM predictions lie within the 68\%~CL regions of this measurement.

% %%%%%%%%%%%%%%%%%%%%%%%
\begin{figure}[!htb]
\begin{center}
\includegraphics[width=0.9\linewidth]{Figures/results/signalstrength/rvrf_125_38.pdf}
\caption{
Result of the 2D likelihood scan for the $\muF$ and $\muV$ signal-strength modifiers.
The solid and dashed contours show the 68\% and 95\% CL regions, respectively.
The cross indicates the best-fit value, and the diamond represents the expected value for the SM Higgs boson.
\label{fig:mu2D}}
\end{center}
\end{figure}
% %%%%%%%%%%%%%%%%%%%%%%%






% OLD text

%\textbf{FIXME: Results shown here used 2017 yield parameterization and categorization but systematics, templates, etc... from 2016 analysis}
%With a 2D fit, the signal strength is measured to be $\mu = 0.98^{+0.11}_{-0.10}$  for a fixed mass hypothesis of $\mH=125~\GeV$. %.09~\GeV$
% (and $X.XX^{+X.XX}_{-X.XX}$ when constraining the Higgs boson mass to $\mH = 125.09\pm0.24\GeV$) 
 %for the inclusive event sample. The results are compared to the expected signal-strength modifiers in Table~\ref{tab:sigstr}.

%{\renewcommand{\arraystretch}{1.2}
%\begin{table}[!hb]
%\begin{center}
%\caption{ Expected and observed signal-strength modifiers. \textbf{FIXME: Numbers to be split by syst/stat uncertainties}
%\label{tab:sigstr}}
%\begin{tabular}{l|c|ccccc}
%   &  Inclusive & $\mu_{\Pg\Pg\PH}$ & $\mu_{\mathrm{VBF}}$ &  $\mu_{\rm VHhad}$ & $\mu_{\rm VHlep}$ & $\mu_{\ttH}$ \\
%\hline
%Expected  & $1.00^{+0.15}_{-0.14}({\rm stat.})^{+0.10}_{-0.09}({\rm sys.})$ & $1.00^{+0.24}_{-0.21}$ & $1.00^{+0.25}_{-0.97}$ & $1.00^{+3.97}_{-1.00}$  & $1.00^{+3.93}_{-1.00}$ & $1.00^{+3.22}_{-1.00}$ \\
%Expected &  $1.00^{+0.17}_{-0.15}$ & $1.00^{+0.21}_{-0.19}$ & $1.00^{+1.11}_{-0.85}$ & $1.00^{+3.84}_{-1.00}$  & $1.00^{+2.86}_{-1.00}$ & $1.00^{+2.50}_{-1.00}$ \\
%Observed  & $XX^{+XX}_{-XX}({\rm stat.})^{+XX}_{XX}({\rm sys.})$ & $XX^{XX}_{XX}$ & $XX^{XX}_{XX}$  & $XX^{+XX}_{-XX}$ & $0.00^{+XX}_{-XX}$ &  $0.00^{+XX}_{-XX}$ \\
%\hline
%\end{tabular}
%\end{center}
%\end{table}
%}

%{\renewcommand{\arraystretch}{1.2}
%\begin{table}[!hb]
%\begin{center}
%\caption{ Expected and observed signal-strength modifiers with 2D scan (with kinematic discriminant).
%\label{tab:sigstr}}
%\resizebox{\textwidth}{!}{\begin{tabular}{l|c|ccccc}
%   &  Inclusive & $\mu_{\Pg\Pg\PH}$ & $\mu_{\mathrm{VBF}}$ &  $\mu_{\rm VH}$ & $\mu_{\ttH}$\\
%\hline
%Expected  & $1.00^{+0.12}_{-0.10}$ & $1.00^{+0.14}_{-0.12}$ & $1.00^{+0.60}_{-0.47}$  & ${1.00}^{+1.09}_{-0.82}$ & $1.00^{+1.17}_{-0.72}$ \\
%Observed  & $0.97^{+0.11}_{-0.10}$ & ${0.67}^{+0.51}_{-0.39}$ & ${0.67}^{+0.51}_{-0.39}$  & ${1.21}^{+1.08}_{-0.85}$ & ${0.33}^{+0.91}_{-0.33}$\\

%\hline
%\end{tabular}}
%\end{center}
%\end{table}
%}


%\begin{figure}[htb]
%\begin{center}
%\includegraphics[width=0.6\linewidth]{Figures/Results/signalstrength/signal_strength_categories_unblind_2017.pdf}
%\caption{Values of $\mu=\sigma/\sigma_{SM}$ for the six categories. 
%The vertical line shows the combined $\mu$, together with its associated $\pm$ 1$\sigma$ uncertainties shown as filled band.  
%The horizontal bars indicate the $\pm$ 1$\sigma$ uncertainties on $\mu$ for the different categories. 
%The uncertainties include both statistical and systematic sources.
%\label{fig:mucat}}
%\end{center}
%\end{figure}

% TO BE ADDED. COMMENTED FOR NOW
%Two signal-strength modifiers $\muF$ and $\muV$ are introduced as scale factors for the fermion and vector-boson induced contribution to the expected SM cross section.
%A two-dimensional fit is performed, profiling the likelihood for all nuisance parameters including the \mH, leading to the measurements of  $\muF=0.98^{+0.13}_{-0.12}$ and $\muV=0.85^{+0.37}_{-0.30}$.
%The 68\% and 95\% CL contours in the ($\muF,\muV$) plane are shown in Fig.~\ref{fig:RVRF}. 
%When constraining the Higgs boson mass to $\mH = 125.09\pm0.24\GeV$ instead of fixing it, the best fit values are $\muF=X.XX$ and $\muV=X.XX$.
%All results for $\mu$, $\muF$ and $\muV$ are reported in Tables~\ref{tab:muvalues_exp} and~\ref{tab:muvalues_obs}.

%\begin{figure}[htb]
%\begin{center}
%    \includegraphics[width=0.6\linewidth]{Figures/results/signalstrength/rvrf.pdf}
%    \caption{ Result of the 2D likelihood scan for the $\muF$ and $\muV$ signal-strength modifiers. 
%   The solid and dashed contours show the 68\% and 95\% CL regions, respectively. 
%    The cross indicates the best-fit values, and the diamond represents the expected values for the SM Higgs boson.
%\label{fig:RVRF}}
%\end{center}
%\end{figure}


%{\renewcommand{\arraystretch}{1.5}
%%%============
%\begin{table}[!hb]
%\begin{center}
%\caption{Expected signal strength values for different measurement setups.
%\label{tab:muvalues_exp}}
%\begin{tabular}{lcccc}
%\hline  
% Categorization  & 1D: ${\cal L}(m_{4l})$ & 2D: ${\cal L}(m_{4l},\KD)$ &   1D: ${\cal L}(m_{4l}')$ & 2D: ${\cal L}(m_{4l}',\KD)$ \\ \hline 
%Inclusive &  $\mu=$1.000$^{+XX}_{-XX}$ & $\mu=$1.000$^{+XX}_{-XX}$ &  $\mu=$1.000$^{+XX}_{-XX}$ & $\mu=$1.000$^{+XX}_{-XX}$ \\ \hline
% &  $\mu=$1.000$^{+XX}_{-XX}$ & $\mu=$1.000$^{+XX}_{-XX}$ &  $\mu=$1.000$^{+XX}_{-XX}$ & $\mu=$1.000$^{+XX}_{-XX}$ \\
%Two categories & $\muV$=1.000$^{+XX}_{-XX}$ & $\muV=$1.000$^{+XX}_{-XX}$ &  $\muV=$1.000$^{+XX}_{-XX}$ & $\muV=$1.000$^{+XX}_{-XX}$ \\
% & $\muF=$1.000$^{+XX}_{-XX}$ & $\muF=$1.000$^{+XX}_{-XX}$ &  $\muF=$1.000$^{+XX}_{-XX}$ & $\muF=$1.000$^{+XX}_{-XX}$ \\
%\hline
%\end{tabular}
%\end{center}
%\end{table}
%%%============
%}
%
%{\renewcommand{\arraystretch}{1.5}
%%%============
%\begin{table}[!hb]
%\begin{center}
%\caption{Expected and observed signal strength values for different measurement setups.
%\label{tab:muvalues_obs}}
%\begin{tabular}{lcccc}
%\hline  
% Categorization  & 1D: ${\cal L}(m_{4l})$ & 2D: ${\cal L}(m_{4l},\KD)$ &   1D: ${\cal L}(m_{4l}')$ & 2D: ${\cal L}(m_{4l}',\KD)$ \\ \hline 
%Inclusive &  $\mu=$X.XX$^{+XX}_{-XX}$ & $\mu=$X.XX$^{+XX}_{-XX}$ &  $\mu=$X.XX$^{+XX}_{-0.XX}$ & $\mu=$X.XX$^{+XX}_{-XX}$ \\ \hline
% &  $\mu=$X.XX$^{+XX}_{-XX}$ & $\mu=$X.XX$^{+XX}_{-XX}$ &  $\mu=$X.XX$^{+XX}_{-XX}$ & $\mu=$X.XX$^{+XX}_{-XX}$ \\
%Two categories & $\muV$=X.XX$^{+XX}_{-XX}$ & $\muV=$X.XX$^{+XX}_{-XX}$ &  $\muV=$X.XX$^{+XX}_{-XX}$ & $\muV=$X.XX$^{+XX}_{-XX}$ \\
% & $\muF=$X.XX$^{+XX}_{-XX}$ & $\muF=$X.XX$^{+XX}_{-XX}$ &  $\muF=$X.XX$^{+XX}_{-XX}$ & $\muF=$X.XX$^{+XX}_{-XX}$ \\
%\hline
%\end{tabular}
%\end{center}
%\end{table}
%%%============
%}

%Signal-strength modifiers for the five main Higgs production modes, $\mu_{\Pg\Pg\PH}$, $\mu_{\mathrm{VBF}}$, $\mu_{\WH}$, $\mu_{\ZH}$ and $\mu_{\ttH}$ are also introduced as scale factors to the expected SM cross section. 
%Similarly, the fits are performed fixing the mass to $\mH = 125~\GeV$ and profiling the likelihood for all nuisance parameters, leading to the results 
%reported in Table~\ref{tab:muprocesses} and 
%illustrated in Fig.~\ref{fig:muprocesses},
% Three models are tested, 
% in Fig.~\ref{fig:muprocesses}(top left) $\mu_{\WH}$ are $\mu_{\ZH}$ are separated, 
% in Fig.~\ref{fig:muprocesses}(top right) $\mu_{\WH}$ are $\mu_{\ZH}$ are grouped in $\mu_{\VH}$, 
% and in Fig.~\ref{fig:muprocesses}(bottom) 
%where $\mu_{\WH}$ and  $\mu_{\ZH}$ are merged to $\mu_{VH}$.
%splitted in the leptonically decay bosons  $\mu_{\WH lep}$/$\mu_{\ZH lep}$ and hadronically decay  bosons $\mu_{\WH lep}$/$\mu_{\ZH lep}$.


%%%============
%\begin{table}[hb]
%\begin{center}
%\caption{Expected and observed results of likelihood scans for the signal strength modifiers corresponding to the five main Higgs boson production modes.
%\label{tab:muprocesses}}
%\begin{tabular}{l|c|c}
%\hline
%   & Expected & Observed  \\
%\hline
%$\mu_{\Pg\Pg\PH}$  & 1.00$^{+0.48}_{-0.43}$ & 1.09$^{+0.47}_{-0.42}$  \\
%$\mu_{\mathrm{VBF}}$  & 1.00$^{+2.51}_{-1.00}$ & 0.00$^{+0.59}_{-0.00}$  \\
%$\mu_{\WH}$  & 1.00$^{+9.20}_{-1.00}$ & 0.00$^{+8.49}_{0.00}$  \\
%$\mu_{\ZH}$  & 1.00$^{+17.96}_{-1.00}$ & 2.48$^{+17.43}_{-2.48}$  \\
%$\mu_{\ttH}$  & 1.00$^{+8.01}_{-1.00}$ & 8.83$^{+16.39}_{-8.83}$ \\
%\hline
%\end{tabular}
%\end{center}
%\end{table}
%%%============

%\begin{figure}[htb]
%\begin{center}
% \includegraphics[width=0.45\linewidth]{Figures/Results/signalstrength/compatibility_process_ZHWH.pdf}
% \includegraphics[width=0.45\linewidth]{Figures/Results/signalstrength/compatibility_process_VH.pdf} \\
%\includegraphics[width=0.80\linewidth]{Figures/results/signalstrength/mu_stage0_hadlep1_obs.pdf}
%\includegraphics[width=0.45\linewidth]{Figures/results/signalstrength/mu_stage0.pdf}
%\includegraphics[width=0.45\linewidth]{Figures/results/signalstrength/mu_stage0_xsec.pdf}
%\caption{Results of likelihood scans for the signal strength modifiers (left) and cross section (right) corresponding to the four main Higgs boson production modes.
%On the left figure, the vertical line shows the combined $\mu$, together with its associated $\pm$ 1$\sigma$ uncertainties shown as filled band.  
%On the right figure, the band of the vertical line shows the theoretical uncertainties on the SM cross sections.
%The horizontal bars indicate the $\pm$ 1$\sigma$ uncertainties on $\mu$ for the different production modes. 
%The uncertainties include both statistical and systematic sources.
%\label{fig:muprocesses}}
%\end{center}
%\end{figure}
 
%
%\clearpage 
%
%%\subsection{Simplified Template Cross Sections}
%%\label{sec:stxs}
%\subsection{Simplified template cross section}
\label{subsec:stxs}

The measurements of the product $\sigma\cdot\mathcal{B}$ of the Higgs boson production cross-section and the branching ratio normalised by the SM expectation,
$(\sigma\cdot\mathcal{B})_{\mathrm{SM}}$, for the stages of production bins defined in Section~\ref{subsec:STXS_Categories} are shown in Fig.~\ref{fig:stxs_0} for the stage 0 and in Fig.~\ref{fig:stxs_1} for the reduce stage 1.2.
The corresponding numerical values are given in Table~\ref{tab:stage0} and Table~\ref{tab:stage1p2}.
In the ratio calculation, the uncertainties in the SM expectation are not taken into account while the theoretical uncertainties which can cause migration of events between the various categories are kept in this measurement.
The correlation matrices are shown in Fig.~\ref{fig:corrmatrix}.
The dominant experimental sources of systematic uncertainty are the same as in the measurement of the signal-strength modifiers, while the dominant theoretical source is the uncertainty in the category migration for the $\ggH$ process.

As for the signal strenght, all measurements are reported at $\mH=125.38$ GeV.%, the best mass obtained by CMS from the combination of $H\rightarrow$ZZ and $H\rightarrow\gamma\gamma$ channels.

%%%%%%%%%%%%%%%%%%%%%%%
\begin{figure}[!htb]
\begin{center}
\includegraphics[width=0.7\linewidth]{Figures/results/stxs/stage0_125_38.pdf}
\caption{
		The ratios between measured cross sections $\sigma\cdot\mathcal{B}$ and the SM predictions $(\sigma\cdot\mathcal{B})_{\mathrm{SM}}$ for the stage 0 production bins.%, with $\mH$ profiled in the fit.
		The gray band around the vertical band shows the theoretical uncertainties on the SM Higgs boson cross section predictions for each of the bin.
		%Cross section values are reported for the best fit mass value $\mH = 125.1~\GeV$.
			\label{fig:stxs_0}
			}
	\end{center}
\end{figure}
%%%%%%%%%%%%%%%%%%%%%%%

%%%%%%%%%%%%%%%%%%%%%%%
\begin{table}[hb]
	\begin{center}
		\caption{
		Best-fit values and $\pm 1\sigma$ uncertainties for the measured cross sections $\sigma\cdot\mathcal{B}$ and the SM predictions $(\sigma\cdot\mathcal{B})_{\mathrm{SM}}$ for the stage 0 production bins.
		The results are obtained with $\mH=125.38$ GeV. % profiled in the fit.
		\label{tab:stage0}
			}
    \renewcommand{\arraystretch}{1.5}
    \begin{tabular}{ccc}
	\hline
	& $\sigma\cdot\mathcal{B}$ & $(\sigma\cdot\mathcal{B})_{\mathrm{SM}}$ \\
	\hline
        {\tt ttH} & $2~^{+13}_{-2}~$fb & $13~^{+18}_{-10}~$fb \\
	{\tt VH-lep} & $41~^{+52}_{-35}~$fb & $26~^{+42}_{-25}~$fb \\
	{\tt qqH} & $61~^{+53}_{-44}~$fb & $122~^{+62}_{-52}~$fb \\
%	{\tt ggH} & $1214~\pm 124~$fb & $1174\pm 123~$fb \\
	{\tt ggH} & $1214~^{+135}_{-125}~$fb & $1192~^{+139}_{-129}~$fb \\
    \hline
    {\tt Inclusive} & $1316~^{+130}_{-122}~$fb & $1366~^{+138}_{-126}~$fb  \\
	\hline
	\end{tabular}
 \end{center}
 \end{table}
 %%%%%%%%%%%%%%%%%%%%%%%

%%%%%%%%%%%%%%%%%%%%%%%
\begin{figure}[!htb]
\begin{center}
\includegraphics[width=0.9\linewidth]{Figures/results/stxs/stage1p2_125_38.pdf}
\caption{
The ratios between measured cross sections $\sigma\cdot\mathcal{B}$ and the SM predictions $(\sigma\cdot\mathcal{B})_{\mathrm{SM}}$ for the merged stage 1.2 production bins, with $\mH=125.38$ GeV. % profiled in the fit.
The gray band around the vertical band shows the theoretical uncertainties on the SM Higgs boson cross section predictions for each of the bin.
\label{fig:stxs_1}
}
\end{center}
\end{figure}
%%%%%%%%%%%%%%%%%%%%%%%

%%%%%%%%%%%%%%%%%%%%%%%
\begin{table}[hb]
	\begin{center}
		\caption{
		Best-fit values and $\pm 1\sigma$ uncertainties for the measured cross sections $\sigma\cdot\mathcal{B}$ and the SM predictions $(\sigma\cdot\mathcal{B})_{\mathrm{SM}}$ for the stage 1.2 production bins.
		The results are obtained with $\mH$ profiled in the fit.
		\label{tab:stage1p2}
			}
    \renewcommand{\arraystretch}{1.5}
    \begin{tabular}{ccc}
	\hline
	& $\sigma\cdot\mathcal{B}$ & $(\sigma\cdot\mathcal{B})_{\mathrm{SM}}$ \\
	\hline
	{\tt ggH-0j/pT[0-10]} & $145~^{+45}_{-40}~$fb & $164~^{+48}_{-42}~$fb \\
	{\tt ggH-0j/pT[10-200]} & $611~^{+98}_{-90}~$fb & $560~^{+98}_{-91}~$fb \\
	{\tt ggH-1j/pT[0-60]} & $214~^{+78}_{-87}~$fb & $177~^{+98}_{-91}~$fb \\
    {\tt ggH-1j/pT[60-120]} & $59~^{+44}_{-53}~$fb & $121~^{+66}_{-55}~$fb \\
	{\tt ggH-1j/pT[120-200]} & $53~^{+25}_{-22}~$fb & $20~^{+21}_{-18}~$fb \\
	{\tt ggH-2j/pT[0-60]} & $0~^{+27}_{-0}~$fb & $35~^{+53}_{-35}~$fb \\
	{\tt ggH-2j/pT[60-120]} & $78~^{+41}_{-37}~$fb & $51~^{+50}_{-42}~$fb \\
	{\tt ggH-2j/pT[120-200]} & $27~^{+22}_{-19}~$fb & $26~^{+27}_{-21}~$fb \\
	{\tt ggH-2j/mJJ>350} & $4~^{+72}_{-4}~$fb & $23~^{+57}_{-0.23}~$fb \\
	{\tt ggH/pT>200} & $7~^{+8}_{-7}~$fb & $15~^{+15}_{-12}~$fb \\
	{\tt qqH-rest} & $11~^{+161}_{-11}~$fb & $71~^{+190}_{-71}~$fb \\
	{\tt qqH-2j/mJJ[60-120]} & $12~^{+30}_{-12}~$fb & $12~^{+69}_{-12}~$fb \\
	{\tt qqH-2j/mJJ[350-700]} & $0~^{+3}_{-0}~$fb & $4~^{+9}_{-4}~$fb \\
	{\tt qqH-2j/mJJ>700} & $15~^{+23}_{-15}~$fb & $10~^{+26}_{-10}~$fb \\
	{\tt qqH-3j/mJJ>350} & $43~^{+30}_{-43}~$fb & $9~^{+35}_{-9}~$fb \\
	{\tt qqH-2j/pT>200} & $0~^{+12}_{-0}~$fb & $15~^{+20}_{-14}~$fb \\
	{\tt VH-lep/pTV[0-150]} & $56~^{+58}_{-40}~$fb & $22~^{+44}_{-22}~$fb \\
	{\tt VH-lep/pTV>150} & $0~^{+10}_{-0}~$fb & $4~^{+16}_{-4}~$fb \\
	{\tt ttH} & $0~^{+12}_{-0}~$fb & $13~^{+18}_{-10}~$fb \\
	\hline
	\end{tabular}
 \end{center}
 \end{table}
 %%%%%%%%%%%%%%%%%%%%%%%

%%%%%%%%%%%%%%%%%%%%%%%
\begin{figure}[!htb]
	\begin{center}
		\includegraphics[height=0.6\linewidth]{Figures/results/stxs/scov_stxs0_125_38.pdf}
		\includegraphics[height=0.6\linewidth]{Figures/results/stxs/scov_stxs1p2_125_38.pdf}
		\caption{The correlation matrices between the measured cross-sections are shown for the stage 0 (left) and the merged stage 1.2 (right).
			\label{fig:corrmatrix}}
	\end{center}
\end{figure}
%%%%%%%%%%%%%%%%%%%%%%%




%In this section we present the results for simplified template cross sections Stage 1.1, a measurement strategy detailed in the CERN Yellow Report 4 of the LHC-HXSWG~\cite{YR4}. The reduced stage 1.1 bins are measured in forms of signal strength modifier. Figure~\ref{fig:stage1p1_mu} shows the results. The covariance matrix of the fitted results are shown in Figure~\ref{fig:cov} 

%%=======
%\begin{figure}[!htb]
%        \vspace*{0.3cm}
%        \begin{center}
%                \includegraphics[width=0.8\textwidth]{Figures/results/stxs/mu_stage1_xsec.pdf}
%                \caption{The ratios between measured cross sections and the SM prediction for stage1.1 bins, the \mH is profiled. The band around the vertical band shows the theoretical uncertainties on the SM cross section predictions for each stage 1.1 bin.}
%                \label{fig:stage1p1_mu}}
%        \end{center}
%\end{figure}
%=======

%=======
%\begin{figure}[!htb]
%        \vspace*{0.3cm}
%        \begin{center}
%                \includegraphics[width=0.8\textwidth]{Figures/results/stxs/scov_stage1.pdf}
%                \caption{Covariance matrix of the fitted signal strength.}
%                \label{fig:cov}}
%        \end{center}
%\end{figure}
%=======
 
\subsection{Yields and distributions}
%\textbf{FIX ME: Distributions are updated but yields are NOT updated}

\section{Yields and distributions}

In this section we summarize the status of the analysis after selection, showing the inputs to the final results, namely the event yields and errors in the full signal region and in a restricted  $\mllll$ range, and the distributions of the main kinematic variables in data and MC. 

\subsection{Signal Region Yields}

The number of candidates observed in data and the expected yields for the backgrounds and Higgs boson signal after the full event selection are reported in Table~\ref{tab:EventYieldsFull} for the full range of $\mllll$. % and in Table~\ref{tab:EventYieldsPeak} for a $118<\mllll<130~\GeV$ mass window around the Higgs boson peak.
%in Table~\ref{tab:EventYieldsPeak} for a $118<\mllll<130~\GeV$ mass window around the Higgs boson peak, 
Table~\ref{tab:EventYieldsPeakCateg} shows the expected and observed yields for each of the 22 event categories, for a $118<\mllll<130~\GeV$ mass window around the Higgs boson peak.

%============
\begin{table}[htb]
	\begin{center}
		\small
		\caption{The number of expected background and signal events 
			and number observed candidates after full analysis selection, for each final state, 
			for the full mass range $\mllll>70~\GeV$, for an integrated luminosity of $\usedLumiABC$.
			Signal and ZZ backgrounds are estimated from Monte Carlo simulation,
			$\cPZ$+X is estimated from data.
			\label{tab:EventYieldsFull}}
			
			\begin{tabular}{|c|c|c|c|c|}
	\hline
	\hline
	\textbf{Channel} & $4\mu$ & $4 e$ & $2e2\mu$ & $4l$ \\
	\hline
	qqZZ     &1414.85        &748.64         &1835.05        &3998.54        \\
	ggZZ     &268.48         &163.45         &399.78         &831.70         \\
	ZX       &112.84         &48.62  &151.71         &313.18         \\
	EW bkg   &15.13  &12.73  &27.83  &55.69  \\
	\hline
	Sum of backgrounds       &1811.30        &973.45         &2414.36        &5199.11        \\
	\hline
	Signal ($m_{H}$ =125 GeV)        &95.27  &46.01  &118.53         &259.82         \\
	\hline
	Total expected   &1906.57       &1019.46        &2532.90        &5458.93         \\
	\hline
	Data     &1970   &1032   &2646   &5648   \\
	\hline
	\hline
\end{tabular}

%			\begin{tabular}{l|c|c|c|c}
%	\hline
%	\hline
%	\textbf{Channel} & $4\Pe$ & $4\Pgm$ & $2\Pe2\Pgm$ & $4\ell$ \\
%	\hline
%	\qqZZ & $333.01^{+ y.y}_{- z.z}$ & $622.20^{+ y.y}_{- z.z}$ & $815.20^{+ y.y}_{- z.z}$ & $1770.41^{+ y.y}_{- z.z}$ \\
%	\ggZZ & $75.11^{+ y.y}_{- z.z}$ & $116.55^{+ y.y}_{- z.z}$ & $176.86^{+ y.y}_{- z.z}$ & $368.52^{+ y.y}_{- z.z}$ \\
%%	\cPZ\ + X & $19.37^{+ y.y}_{- z.z}$ & $50.75^{+ y.y}_{- z.z}$ & $64.84^{+ y.y}_{- z.z}$ & $134.96^{+ y.y}_{- z.z}$ \\
%	\hline
%	Sum of backgrounds & $427.50^{+ y.y}_{- z.z}$ & $789.50^{+ y.y}_{- z.z}$ & $1056.89^{+ y.y}_{- z.z}$ & $2273.89^{+ y.y}_{- z.z}$ \\
%	\hline
%	Signal ($\mH=125~\GeV$) & $19.58^{+ y.y}_{- z.z}$ & $40.83^{+ y.y}_{- z.z}$ & $50.68^{+ y.y}_{- z.z}$ & $111.09^{+ y.y}_{- z.z}$ \\
%	\hline
%	Total expected & $447.08^{+ y.y}_{- z.z}$ & $830.33^{+ y.y}_{- z.z}$ & $1107.57^{+ y.y}_{- z.z}$ & $2384.98^{+ y.y}_{- z.z}$ \\
%	\hline
%	Observed & 462 & 850 & 1130 & 2442 \\
%	\hline
%	\hline
%\end{tabular}

	%		\begin{tabular}{l|c|c|c|c}
%				\hline
		%		\hline
			%	\textbf{Channel} & $4\Pe$ & $4\Pgm$ & $2\Pe2\Pgm$ & $4\ell$ \\
				%\hline
				%\qqZZ & $368.00^{+ y.y}_{- z.z}$ & $646.66^{+ y.y}_{- z.z}$ & $871.70^{+ y.y}_{- z.z}$ & $1886.35^{+ y.y}_{- z.z}$ \\
				%\ggZZ & $82.36^{+ y.y}_{- z.z}$ & $122.98^{+ y.y}_{- z.z}$ & $190.37^{+ y.y}_{- z.z}$ & $395.71^{+ y.y}_{- z.z}$ \\
				%\cPZ\ + X & $22.27^{+ y.y}_{- z.z}$ & $51.41^{+ y.y}_{- z.z}$ & $63.90^{+ y.y}_{- z.z}$ & $137.58^{+ y.y}_{- z.z}$ \\
				%\hline
				%Sum of backgrounds & $472.63^{+ y.y}_{- z.z}$ & $821.05^{+ y.y}_{- z.z}$ & $1125.97^{+ y.y}_{- z.z}$ & $2419.65^{+ y.y}_{- z.z}$ \\
				%\hline
				%Signal ($\mH=125~\GeV$) & $21.68^{+ y.y}_{- z.z}$ & $42.57^{+ y.y}_{- z.z}$ & $54.32^{+ y.y}_{- z.z}$ & $118.57^{+ y.y}_{- z.z}$ \\
				%\hline
				%Total expected & $494.31^{+ y.y}_{- z.z}$ & $863.61^{+ y.y}_{- z.z}$ & $1180.29^{+ y.y}_{- z.z}$ & $2538.22^{+ y.y}_{- z.z}$ \\
				%\hline
				%Observed & 466 & 845 & 1121 & 2432\\
				%\hline
				%\hline
			%\end{tabular}
	\end{center}
\end{table}
%============


%\begin{table}[htb]
%	\begin{center}
%		\small
%		\caption{\textbf{[FIXME] To be updated with new numbers!}The number of expected background and signal events and number of observed candidates after full analysis selection, for each event category, for the mass range $118<\mllll<130~\GeV$, for an integrated luminosity of $\usedLumiABC$.
%			The yields are given for the different production modes.
%			Signal and ZZ backgrounds are estimated from Monte Carlo simulation, 
%			$\cPZ$+X is estimated from data.
%			\label{tab:EventYieldsPeak}}
		%\resizebox{\textwidth}{!}
		{
		%	\begin{tabular}{l|c|c|c|c}
%			\begin{tabular}{l|c|c|c}
%				\hline
%				\hline
%				\textbf{Channel} & $4\Pe$ & $4\Pgm$ & $2\Pe2\Pgm$ & $4\ell$ \\
%				\textbf{Channel} & $4\Pgm$ & $4\Pe$ & $2\Pe2\Pgm$  \\
%				\hline
%				Signal ($m_{H}$ =125 GeV)  &37.09  &16.35  &44.16   \\
%				 qqZZ     &14.10  &5.33   &16.43   \\
%				 ggZZ     &1.30   &0.63   &1.15    \\
%				 ZX       &6.53   &1.72   &7.22    \\
%				 Total expected   &59.02 &24.03  &68.96   \\
%				 Data     &50     &23     &63      \\
				%\qqZZ & $5.77^{+ y.y}_{- z.z}$ & $15.49^{+ y.y}_{- z.z}$ & $17.78^{+ y.y}_{- z.z}$ & $39.04^{+ y.y}_{- z.z}$ \\
				%\ggZZ & $0.71^{+ y.y}_{- z.z}$ & $1.54^{+ y.y}_{- z.z}$ & $1.35^{+ y.y}_{- z.z}$ & $3.60^{+ y.y}_{- z.z}$ \\
				%\cPZ\ + X & $2.04^{+ y.y}_{- z.z}$ & $7.03^{+ y.y}_{- z.z}$ & $7.34^{+ y.y}_{- z.z}$ & $16.41^{+ y.y}_{- z.z}$ \\
				%\hline
				%Sum of backgrounds & $8.52^{+ y.y}_{- z.z}$ & $24.07^{+ y.y}_{- z.z}$ & $26.47^{+ y.y}_{- z.z}$ & $59.05^{+ y.y}_{- z.z}$ \\
				%\hline
				%Signal ($\mH=125~\GeV$) & $18.32^{+ y.y}_{- z.z}$ & $38.68^{+ y.y}_{- z.z}$ & $47.67^{+ y.y}_{- z.z}$ & $104.68^{+ y.y}_{- z.z}$ \\
				%\hline
				%Total expected & $26.84^{+ y.y}_{- z.z}$ & $62.75^{+ y.y}_{- z.z}$ & $74.14^{+ y.y}_{- z.z}$ & $163.73^{+ y.y}_{- z.z}$ \\
				%\hline
				%Observed & 22 & 50 & 61 & 133 \\
%				\hline
%				\hline
%		\end{tabular}}
%	\end{center}
%\end{table}


%============
%\begin{table}[htb]
%	\begin{center}
%		\caption{The number of expected background and signal events and number of observed candidates after full analysis selection,
%		for each event category, for the mass range $118<\mllll<130~\GeV$, for an integrated luminosity of $\usedLumiABC$.
%		The yields are given for the different production modes.
%		Signal, $\cPZ\cPZ$ and rare electroweak backgrounds are estimated from Monte Carlo simulation, $\cPZ$+X is estimated from data.
%		\label{tab:EventYieldsPeakCateg}}
%		\cmsTable
%		{
%				\begin{tabular}{ccccccccccccccc}
%				\hline
%				Event & \multicolumn{7}{|c}{Signal} & \multicolumn{4}{|c}{Background} &  \multicolumn{2}{|c|}{Expected} & Observed \\
%
%				 category & \multicolumn{1}{|c}{$\ggH$} &\VBF    &\WH    &\ZH    &\ttH   &\bbH   & \tH  & \multicolumn{1}{|c}{$\qqZZ$} &$\ggZZ$  &  EWK &  $\cPZ$+X & \multicolumn{1}{|c}{signal} &  \multicolumn{1}{c|}{total} &  \\
%				\hline
%				Untagged-0j-$\pt^{4\ell}[0,10]$ & 25.75 & 0.08 & 0.02 & 0.03 & 0.00 & 0.14 & 0.00 & 26.95 & 1.08 & 0.00 & 1.09 & 26.01 & 55.14 & 59
%				\\
%				Untagged-0j-$\pt^{4\ell}[10,200]$ & 88.86 & 1.56 & 0.53 & 1.09 & 0.00 & 0.93 & 0.00 & 36.40 & 4.10 & 0.06 & 14.78 & 92.97 & 148.31 & 163
%				\\
%				Untagged-1j-$\pt^{4\ell}[0,60]$ & 24.74 & 1.39 & 0.5 & 0.5 & 0.01 & 0.42 & 0.01 & 9.46 & 1.10 & 0.13 & 5.52 & 27.58 & 43.78 & 40
%				\\
%				Untagged-1j-$\pt^{4\ell}[60,120]$ & 12.59 & 1.22 & 0.46 & 0.59 & 0.01 & 0.10 & 0.01 & 2.88 & 0.28 & 0.02 & 3.12 & 14.98 & 21.29 & 15
%				\\
%				Untagged-1j-$\pt^{4\ell}[120,200]$ & 3.33 & 0.56 & 0.16 & 0.39 & 0.01 & 0.02 & 0.00 & 0.39 & 0.06 & 0.00 & 0.48 & 4.47 & 5.40 & 9
%				\\
%				Untagged-2j-$\pt^{4\ell}[0,60]$ & 3.15 & 0.26 & 0.13 & 0.16 & 0.06 & 0.08 & 0.02 & 0.77 & 0.12 & 0.06 & 2.05 & 3.86 & 6.86 & 7
%				\\
%				Untagged-2j-$\pt^{4\ell}[60,120]$ & 4.85 & 0.51 & 0.21 & 0.28 & 0.09 & 0.04 & 0.02 & 0.78 & 0.10 & 0.05 & 1.80 & 6.01 & 8.75 & 10
%				\\
%				Untagged-2j-$\pt^{4\ell}[120,200]$ & 2.86 & 0.38 & 0.15 & 0.24 & 0.07 & 0.01 & 0.01 & 0.26 & 0.05 & 0.02 & 0.43 & 3.72 & 4.48 & 5
%				\\
%				Untagged-$\pt^{4\ell}\gt200$ & 2.64 & 0.59 & 0.19 & 0.32 & 0.07 & 0.01 & 0.02 & 0.16 & 0.07 & 0.05 & 0.22 & 3.84 & 4.34 & 2
%				\\
%				Untagged-2j-$m_{jj}\gt350$ & 0.72 & 0.15 & 0.06 & 0.06 & 0.04 & 0.01 & 0.01 & 0.13 & 0.02 & 0.01 & 0.59 & 1.06 & 1.81 & 3
%				\\
%				$\VBF$-1jet-tagged & 14.38 & 3.04 & 0.2 & 0.25 & 0.00 & 0.12 & 0.01 & 2.39 & 0.53 & 0.00 & 0.99 & 18.01 & 21.92 & 20
%				\\
%				$\VBF$-2jet-tagged-$m_{jj}[350,700]$ & 0.77 & 1.10 & 0.01 & 0.01 & 0.00 & 0.01 & 0.00 & 0.06 & 0.02 & 0.00 & 0.05 & 1.91 & 2.04 & 2
%				\\
%				$\VBF$-2jet-tagged-$m_{jj}\gt700$ & 0.41 & 1.83 & 0.0 & 0.0 & 0.00 & 0.00 & 0.00 & 0.02 & 0.02 & 0.00 & 0.02 & 2.25 & 2.32 & 1
%				\\
%				$\VBF$-3jet-tagged-$m_{jj}\gt350$ & 2.34 & 2.18 & 0.07 & 0.1 & 0.03 & 0.03 & 0.04 & 0.23 & 0.07 & 0.01 & 1.05 & 4.78 & 6.14 & 10
%				\\
%				$\VBF$-2jet-tagged-$\pt^{4\ell}\gt200$ & 0.42 & 0.75 & 0.01 & 0.01 & 0.01 & 0.00 & 0.01 & 0.01 & 0.01 & 0.00 & 0.03 & 1.21 & 1.26 & 0
%				\\
%				VBF-rest & 2.27 & 0.87 & 0.11 & 0.12 & 0.03 & 0.03 & 0.01 & 0.32 & 0.07 & 0.02 & 0.76 & 3.46 & 4.64 & 2
%				\\
%				$\VH$-hadronic-tagged-$m_{jj}[60,120]$ & 3.88 & 0.23 & 1.02 & 1.82 & 0.11 & 0.06 & 0.02 & 0.62 & 0.08 & 0.03 & 1.19 & 7.14 & 9.07 & 7
%				\\
%				VH-rest & 0.56 & 0.03 & 0.09 & 0.1 & 0.02 & 0.00 & 0.00 & 0.08 & 0.01 & 0.00 & 0.12 & 0.81 & 1.03 & 0
%				\\
%				$\VH$-leptonic-tagged-$\pt^{4\ell}[0,150]$ & 0.21 & 0.03 & 0.7 & 0.32 & 0.08 & 0.02 & 0.02 & 0.75 & 0.13 & 0.01 & 0.37 & 1.38 & 2.65 & 3
%				\\
%				$\VH$-leptonic-tagged-$\pt^{4\ell}\gt150$ & 0.02 & 0.01 & 0.2 & 0.13 & 0.04 & 0.00 & 0.01 & 0.01 & 0.00 & 0.00 & 0.03 & 0.40 & 0.45 & 0
%				\\
%				$\ttH$-leptonic-tagged & 0.02 & 0.00 & 0.02 & 0.02 & 0.59 & 0.00 & 0.03 & 0.03 & 0.00 & 0.02 & 0.10 & 0.69 & 0.86 & 0
%				\\
%				$\ttH$-hadronic-tagged & 0.17 & 0.05 & 0.03 & 0.08 & 0.81 & 0.01 & 0.03 & 0.01 & 0.00 & 0.05 & 0.45 & 1.17 & 1.69 & 2
%				\\
%				\hline
%\end{tabular}}
%\end{center}
%\end{table}
%============


%\begin{table}[htb]
%	\begin{center}
%		\small
%		\caption{\textbf{[FIXME] To be updated with new numbers!}The number of expected background and signal events and number of observed candidates after full analysis selection, for each event category, for the mass range $118<\mllll<130~\GeV$, for an integrated luminosity of \usedLumiABC.
%			The yields are given for the different production modes.
%			Signal and ZZ backgrounds are estimated from Monte Carlo simulation, 
%			$\cPZ$+X is estimated from data.
%			\label{tab:EventYieldsPeakCateg}}
%		\resizebox{\textwidth}{!}
%		{
		%	\begin{tabular}{c|cccccccccc|ccc|c|c}
	%ù			\begin{tabular}{cccccccc|ccc|cc|c}
	%			\hline
	%			\hline
	%			Event & \multicolumn{7}{c|}{Signal} & \multicolumn{3}{c|}{Background} &  \multicolumn{2}{c|}{Expected} & Observed \\
	%			category             & $\ggH$ &VBF    &WH    &ZH    &ttH   &bbH   &tqH   &$\qqZZ$  &$\ggZZ$   &$\cPZ$ + X  &signal&total& \\
	%			\hline
	%			ggH-0j/pT[0,10]      &25.31  &0.08  &0.02  &0.02  &0.00  &0.14  &0.00  &26.46  &0.97  &1.19    &25.57  &54.18   &61.00   \\
	%			ggH-0j/pT[10-200]    &86.80  &1.69  &0.54  &0.86  &0.00  &0.90  &0.00  &35.42  &3.79  &15.48   &90.80  &145.49  &153.00  \\
	%			ggH-1j/pT[0-60]      &26.24  &1.43  &0.50  &0.45  &0.01  &0.43  &0.01  &10.26  &1.19  &5.54    &29.06  &46.05   &40.00   \\
	%			ggH-1j/pT[60-120]    &12.35  &1.24  &0.45  &0.47  &0.01  &0.10  &0.01  &2.76   &0.16  &3.21    &14.63  &20.76   &17.00   \\
	%			ggH-1j/pT[120-200]   &3.31   &0.62  &0.17  &0.26  &0.00  &0.02  &0.00  &0.38   &0.00  &0.52    &4.38   &5.28    &6.00    \\
	%			ggH-2j/pT[0-60]      &3.68   &0.29  &0.14  &0.14  &0.06  &0.09  &0.02  &0.97   &0.15  &2.07    &4.42   &7.60    &9.00    \\
	%			ggH-2j/pT[60-120]    &5.17   &0.54  &0.22  &0.22  &0.09  &0.04  &0.02  &0.84   &0.07  &1.86    &6.30   &9.06    &12.00   \\
	%			ggH-2j/pT[120-200]   &2.90   &0.40  &0.15  &0.17  &0.07  &0.01  &0.02  &0.26   &0.00  &0.40    &3.71   &4.37    &5.00    \\
	%			ggH/pT$>$200         &2.72   &0.65  &0.21  &0.24  &0.06  &0.01  &0.02  &0.16   &0.00  &0.21    &3.91   &4.28    &2.00    \\
	%			ggH-2j/mJJ$>$350     &0.82   &0.17  &0.06  &0.05  &0.04  &0.01  &0.01  &0.16   &0.02  &0.65    &1.16   &1.98    &3.00    \\
	%			VBF-1j               &14.17  &2.94  &0.20  &0.18  &0.00  &0.12  &0.01  &2.37   &0.43  &1.05    &17.61  &21.46   &20.00   \\
	%			VBF-2j/mJJ[350,700]  &0.80   &1.11  &0.01  &0.01  &0.00  &0.01  &0.00  &0.08   &0.02  &0.04    &1.95   &2.09    &2.00    \\
	%			VBF-2j/mJJ$>$700     &0.43   &1.80  &0.00  &0.00  &0.00  &0.00  &0.00  &0.02   &0.01  &0.03    &2.25   &2.31    &2.00    \\
	%			VBF-3j/mJJ$>$350     &2.43   &2.15  &0.06  &0.07  &0.02  &0.03  &0.05  &0.24   &0.06  &0.96    &4.81   &6.07    &6.00    \\
	%			VBF-2j/pT$>$200      &0.42   &0.76  &0.01  &0.01  &0.01  &0.00  &0.01  &0.01   &0.00  &0.03    &1.22   &1.26    &0.00    \\
	%			VBF-rest             &2.40   &0.87  &0.11  &0.10  &0.03  &0.04  &0.01  &0.34   &0.06  &0.74    &3.56   &4.70    &2.00    \\
	%			VH-lep/pTV[0-150]    &0.24   &0.04  &0.71  &0.25  &0.08  &0.02  &0.02  &0.82   &0.14  &0.40    &1.37   &2.72    &5.00    \\
	%			VH-lep/pTV$>$150     &0.02   &0.01  &0.21  &0.08  &0.04  &0.00  &0.01  &0.01   &0.00  &0.02    &0.36   &0.40    &0.00    \\
	%			VH-had/mJJ[60-120]   &4.11   &0.25  &1.01  &1.20  &0.11  &0.07  &0.02  &0.70   &0.05  &1.36    &6.77   &8.89    &8.00    \\
	%			VH-rest              &0.56   &0.04  &0.08  &0.07  &0.03  &0.00  &0.00  &0.08   &0.00  &0.15    &0.77   &1.01    &1.00    \\
	%			ttH-had              &0.19   &0.05  &0.03  &0.06  &0.82  &0.01  &0.03  &0.01   &0.00  &0.45    &1.19   &1.66    &2.00    \\
	%			ttH-lep              &0.02   &0.00  &0.02  &0.02  &0.60  &0.00  &0.03  &0.03   &0.00  &0.12    &0.70   &0.85    &0.00    \\
	%			\hline
	%			\hline
	%	\end{tabular}}
	%\end{center}
%\end{table}

\subsection{Signal Region Distributions}

The reconstructed four-lepton invariant mass distribution is shown in Figure~\ref{fig:Mass4lC} for the full dataset, and compared to expectations from the SM backgrounds, first for the full mass range, and then zooming on the low-mass range and high-mass range. 
In Figure~\ref{fig:Mass4l-2C}, the same distributions are shown split by final state ($4e$, $4\mu$, and $2e2\mu$), for the two same mass ranges.
In Figure~\ref{fig:Mass4l-3C}, they are split by event category, for the low-mass range.
The SM background distributions are obtained combining the rate normalization from data-driven methods and knowledge on shapes taken from the MC samples. \\
%\textbf{FIXME: 2016 MC is used.} \\
%\textbf{Z+X estimation is scaled from 2016.}.\\
%\textbf{(low mass) Signal region is Blinded, as well as high mass points} \\

%=============
\begin{figure}[!htb]
	\vspace*{0.3cm}
	\begin{center}
		\includegraphics[width=0.85\textwidth]{Figures/KinDistr/M4lMain_Unblinded_4l_InclusiveRun2.png} \\ %a_M4l__4l_inclusive_.pdf}\\
		\includegraphics[width=0.45\textwidth]{Figures/KinDistr/M4lMainZoomed_Unblinded_4l_InclusiveRun2.png} %a_M4l_70170_4l_inclusive_.pdf}
		\includegraphics[width=0.45\textwidth]{Figures/KinDistr/M4lMainHighMass_Unblinded_4l_InclusiveRun2.png} %a_M4l_above170_4l_inclusive_.pdf}
		\caption{Distribution of the four-lepton reconstructed invariant mass $\mllll$ in the full mass range (top) and the low-mass range (bottom left) and high-mass range (bottom right). Points with error bars represent the data and stacked histograms represent expected distributions. The $125~\GeV$ Higgs boson signal and the $\cPZ\cPZ$ backgrounds are normalized to the SM expectation, the $\cPZ$+X background to the estimation from data.
			%what about high mass range??
			%No event is observed with $\mllll>1\TeV$.
			\label{fig:Mass4lC}}
	\end{center}
\end{figure}
%=======

%=============
\begin{figure}[!htb]
	\vspace*{0.3cm}
	\begin{center}
		\includegraphics[width=0.56\textwidth]{Figures/KinDistr/M4lMain_Unblinded_4e_InclusiveRun2.png}
		\includegraphics[width=0.43\textwidth]{Figures/KinDistr/M4lMainZoomed_Unblinded_4e_InclusiveRun2.png}\\
		\includegraphics[width=0.56\textwidth]{Figures/KinDistr/M4lMain_Unblinded_4mu_InclusiveRun2.png}
		\includegraphics[width=0.43\textwidth]{Figures/KinDistr/M4lMainZoomed_Unblinded_4mu_InclusiveRun2.png}\\
		\includegraphics[width=0.56\textwidth]{Figures/KinDistr/M4lMain_Unblinded_2e2mu_InclusiveRun2.png}
		\includegraphics[width=0.43\textwidth]{Figures/KinDistr/M4lMainZoomed_Unblinded_2e2mu_InclusiveRun2.png}\\
		\caption{ Distribution of the four-lepton reconstructed mass in several sub-channels: $4e$ (top), $4\mu$ (middle), $2e2\mu$ for the low-mass range (bottom) for the full mass range (left) and the low-mass range (right).
			\label{fig:Mass4l-2C}}
	\end{center}
\end{figure}
%==================

%=============
%\begin{figure}[!htb]
%	\vspace*{0.3cm}
%	\begin{center}
%		\subfigure[]{ \includegraphics[width=0.45\textwidth]{Figures/KinDistr/M4lMain_Unblinded_4l_UnTaggedRun2.png}} 
%		\subfigure[]{ \includegraphics[width=0.45\textwidth]{Figures/KinDistr/M4lMain_Unblinded_4l_VBF1jTaggedRun2.png}} \\
%		\subfigure[]{ \includegraphics[width=0.45\textwidth]{Figures/KinDistr/M4lMain_Unblinded_4l_VBF2jTaggedRun2.png}} 
%		\subfigure[]{ \includegraphics[width=0.45\textwidth]{Figures/KinDistr/M4lMain_Unblinded_4l_VHHadrTaggedRun2.png}} \\
%		\subfigure[]{ \includegraphics[width=0.33\textwidth]{Figures/KinDistr/M4lMain_Unblinded_4l_VHLeptTaggedRun2.png}} 
%		%\subfigure[]{ \includegraphics[width=0.33\textwidth]{Figures/KinDistr/M4lMain_Blinded_4l_VHMETTagged.pdf}} \\
%		\subfigure[]{ \includegraphics[width=0.33\textwidth]{Figures/KinDistr/M4lMain_Unblinded_4l_ttHHadrTaggedRun2.png}} 
%		\subfigure[]{ \includegraphics[width=0.33\textwidth]{Figures/KinDistr/M4lMain_Unblinded_4l_ttHLeptTaggedRun2.png}} \\
%		\caption{ Distribution of the four-lepton reconstructed mass in the seven event categories for the low-mass range.
%			(a) untagged category (b) VBF-1jet-tagged category (c) VBF-2jet-tagged category (d) VH-hadronic-tagged category (e) VH-leptonic-tagged category (f) \ttH-hadronic-tagged category (g) \ttH-leptonic-tagged category.
%			%(f) VH-MET-tagged category (g) \ttH-hadronic-tagged category (h) \ttH-leptonic-tagged category.
%			\label{fig:Mass4l-3C}}
%	\end{center}
%\end{figure}
%%=======


The reconstructed dilepton invariant masses selected as Z$_1$ and Z$_2$ are shown in Figures~\ref{fig:MZ1C} together with their correlation, both full range of of $\mllll$ and focusing on a $118<\mllll<130~\GeV$ mass window around the Higgs boson peak.
%and~\ref{fig:MZ2} in the full mass range.
%with their correlation, both in the full range of $\mllll$ and focusing on a $118<\mllll<130~\GeV$ mass window around the Higgs boson peak.
Similarly, the decay discriminant $\KD$, $\DbkgVBFdec$ and $\DbkgVHdec$ are shown in Fig.~\ref{fig:KDvsM4lC},~\ref{fig:KDVBFsvsM4lC} and~\ref{fig:KDVHsvsM4lC} in this window. % in these two windows, as well as its correlation with the four-lepton invariant mass.
The four production mechanism discriminant $\DMeVbfjj$, $\DMeVbfj$, $\DMeWh$, and $\DMeZh$ are shown in Fig.~\ref{fig:DprodC} in the $118<\mllll<130~\GeV$ mass window, for events with at least two selected jets (except $\DMeVbfj$ which is for events with exactly one selected jet). 
Their correlations with $\mllll$ are shown in Fig.~\ref{fig:Dprod-corrC}. 

%=============
\begin{figure}[!htb]
	\vspace*{0.3cm}
	\begin{center}
		%\includegraphics[width=0.450\textwidth]{Figures/KinDistr/2018/MZ1_Unblinded_4l_Inclusive.pdf} %fb_MZ1_4l_inclusive_.pdf}
		\includegraphics[width=0.450\textwidth]{Figures/KinDistr/MZ1_M4L118130_Unblinded_4l_InclusiveRun2.png} \\ %fb_MZ1_4l_inclusive_.pdf}
		%\includegraphics[width=0.450\textwidth]{Figures/KinDistr/2018/MZ2_Unblinded_4l_Inclusive.pdf} %fb_MZ1_4l_inclusive_.pdf}
		\includegraphics[width=0.450\textwidth]{Figures/KinDistr/MZ2_M4L118130_Unblinded_4l_InclusiveRun2.png} \\ %fb_MZ1_4l_inclusive_.pdf}
		%\includegraphics[width=0.45\textwidth]{Figures/KinDistr/2018/MZ1vsMZ2_Unblinded_Inclusive.pdf}
%		\includegraphics[width=0.45\textwidth]{Figures/KinDistr/MZ1vsMZ2_M4L118130_Unblinded_InclusiveRun2.png}
%		\caption{%\textb{FIXME: Add low mass plots after unblinding.}
			%Distribution of the $\cPZ_1$ (top), $\cPZ_2$ (middle) and $\cPZ_1$ vs $\cPZ_2$ (bottom) reconstructed invariant masses for the full mass range (left) and the low mass ($118<\mllll<130~\GeV$) range (right). 
			Distribution of the $\cPZ_1$ (left) and  $\cPZ_2$ (right) reconstructed invariant masses for the full mass range (left) and the low mass ($118<\mllll<130~\GeV$) range (right). 
			%Distribution of the $\cPZ_1$  reconstructed invariant masses for the full mass range (blinded), for all final states (top left), and for 4e (top right), 4mu (bottom left) and 2e2mu (bottom right) separately. 
			The stacked histograms and the gray scale represent expected distributions, and points represent the data. The $125~\GeV$ Higgs boson signal and the $\cPZ\cPZ$ backgrounds are normalized to the SM expectation, the $\cPZ$+X background to the estimation from data.
			%Distribution of the $\cPZ_1$ (left) and $\cPZ_2$ (center) reconstructed invariant masses and correlation between the two (right), for the full mass range (top row) and the mass region $118<\mllll<130~\GeV$ (bottom row). The stacked histograms and the gray scale represent expected distributions, and points represent the data. The $125~\GeV$ Higgs boson signal and the $\cPZ\cPZ$ backgrounds are normalized to the SM expectation, the $\cPZ$+X background to the estimation from data.
			\label{fig:MZ1C}}
	\end{center}
\end{figure}
%=======


%=======
%\begin{figure}[!htb]
%	\vspace*{0.3cm}
%	\begin{center}
%	  \includegraphics[width=0.455\textwidth]{Figures/KinDistr/KD_Unblinded_4l_InclusiveRun2.png} %d_KD_4l_inclusive_.pdf}KD_Blinded_4l_Inclusive.pdf
%		\includegraphics[width=0.455\textwidth]{Figures/KinDistr/KD_M4L118130_Unblinded_4l_InclusiveRun2.png} \\
%		%\includegraphics[width=0.455\textwidth]{Figures/KinDistr/2018/KDvsM4l_Unblinded_Inclusive.pdf}
%		\includegraphics[width=0.455\textwidth]{Figures/KinDistr/KDvsM4lZoomed_Unblinded_all_categoriesRun2.png} 
%                \includegraphics[width=0.455\textwidth]{Figures/KinDistr/KDvsM4lHighMass_Unblinded_InclusiveRun2.png} \\
%		
%		\caption{
%		%\textb{FIXME: Add low mass plots after unblinding.}
%			%One-dimensional distribution of the kinematic discriminant $\KD$ in the full mass range (blinded), for all final states (top left) and for 4e  (top right), 4mu (bottom left) and 2e2mu (bottom right) separately.  Points with error bars represent the data and stacked histograms represent expected distributions. The $125~\GeV$ Higgs boson signal and the $\cPZ\cPZ$ backgrounds are normalized to the SM expectation, the $\cPZ$+X background to the estimation from data.
%			Top row: One-dimensional distribution of the kinematic discriminant $\KD$ in the full mass range (left) and in the mass region $118<\mllll<130~\GeV$ (right). Points with error bars represent the data and stacked histograms represent expected distributions. The $125~\GeV$ Higgs boson signal and the $\cPZ\cPZ$ backgrounds are normalized to the SM expectation, the $\cPZ$+X background to the estimation from data.
%			Bottom row: Distribution of the kinematic discriminant $\KD$ versus the four-lepton reconstructed mass $\mllll$ in the low-mass region (left) and in the high-mass region (right). The gray scale represents the expected relative density of $\cPZ\cPZ$ background plus Higgs boson signal for $\mH=125~\GeV$. The points show the data and the horizontal bars represent the measured mass uncertainties. 
%			\label{fig:KDvsM4lC}}
%	\end{center}
%\end{figure}
%%=======
%
%%=======
%\begin{figure}[!htb]
%	\vspace*{0.3cm}
%	\begin{center}
%		\includegraphics[width=0.455\textwidth]{Figures/KinDistr/DVBFDEC_Unblinded_4l_VBF2jTaggedRun2.png} %d_KD_4l_inclusive_.pdf}KD_Blinded_4l_Inclusive.pdf
%		\includegraphics[width=0.455\textwidth]{Figures/KinDistr/DVBFDEC_M4L118130_Unblinded_4l_VBF2jTaggedRun2.png} \\
%		\includegraphics[width=0.455\textwidth]{Figures/KinDistr/DVBFDECvsM4lZoomed_Unblinded_all_categoriesRun2.png}
%		%\includegraphics[width=0.455\textwidth]{Figures/KinDistr/KDvsM4lZoomed_Unblinded_all_categories.pdf} \\
%		\caption{ %\textb{FIXME: Add low mass plots after unblinding.}
%			%One-dimensional distribution of the kinematic discriminant $\KD$ in the full mass range (blinded), for all final states (top left) and for 4e  (top right), 4mu (bottom left) and 2e2mu (bottom right) separately.  Points with error bars represent the data and stacked histograms represent expected distributions. The $125~\GeV$ Higgs boson signal and the $\cPZ\cPZ$ backgrounds are normalized to the SM expectation, the $\cPZ$+X background to the estimation from data.
%			Top row: One-dimensional distribution of the kinematic discriminant $\DbkgVBFdec$ in the full mass range (left) and in the mass region $118<\mllll<130~\GeV$ (right). Points with error bars represent the data and stacked histograms represent expected distributions. The $125~\GeV$ Higgs boson signal and the $\cPZ\cPZ$ backgrounds are normalized to the SM expectation, the $\cPZ$+X background to the estimation from data.
%			Bottom row: Distribution of the kinematic discriminant $\DbkgVBFdec$ versus the four-lepton reconstructed mass $\mllll$ in the low-mass region.
%			% (left) and in the high-mass region (right). 
%			The gray scale represents the expected relative density of $\cPZ\cPZ$ background plus Higgs boson signal for $\mH=125~\GeV$. The points show the data and the horizontal bars represent the measured mass uncertainties. 
%			\label{fig:KDVBFsvsM4lC}}
%	\end{center}
%\end{figure}
%%=======
%
%%=======
%\begin{figure}[!htb]
%	\vspace*{0.3cm}
%	\begin{center}
%	  \includegraphics[width=0.455\textwidth]{Figures/KinDistr/DVHDEC_Unblinded_4l_VHHadrTaggedRun2.png} 
%		\includegraphics[width=0.455\textwidth]{Figures/KinDistr/DVHDEC_M4L118130_Unblinded_4l_VHHadrTaggedRun2.png} \\
%		\includegraphics[width=0.455\textwidth]{Figures/KinDistr/DVHDECvsM4lZoomed_Unblinded_all_categoriesRun2.png}
%		%\includegraphics[width=0.455\textwidth]{Figures/KinDistr/KDvsM4lZoomed_Unblinded_all_categories.pdf} \\
%		\caption{ %\textb{FIXME: Add low mass plots after unblinding.}
%			%One-dimensional distribution of the kinematic discriminant $\KD$ in the full mass range (blinded), for all final states (top left) and for 4e  (top right), 4mu (bottom left) and 2e2mu (bottom right) separately.  Points with error bars represent the data and stacked histograms represent expected distributions. The $125~\GeV$ Higgs boson signal and the $\cPZ\cPZ$ backgrounds are normalized to the SM expectation, the $\cPZ$+X background to the estimation from data.
%			Top row: One-dimensional distribution of the kinematic discriminant $\DbkgVHdec$ in the full mass range (left) and in the mass region $118<\mllll<130~\GeV$ (right). Points with error bars represent the data and stacked histograms represent expected distributions. The $125~\GeV$ Higgs boson signal and the $\cPZ\cPZ$ backgrounds are normalized to the SM expectation, the $\cPZ$+X background to the estimation from data.
%			Bottom row: Distribution of the kinematic discriminant $\DbkgVHdec$ versus the four-lepton reconstructed mass $\mllll$ in the low-mass region.
%			% (left) and in the high-mass region (right). 
%			The gray scale represents the expected relative density of $\cPZ\cPZ$ background plus Higgs boson signal for $\mH=125~\GeV$. The points show the data and the horizontal bars represent the measured mass uncertainties. 
%			\label{fig:KDVHsvsM4lC}}
%	\end{center}
%\end{figure}
%=======

%=======
%\begin{figure}[!htb]
%\vspace*{0.3cm}
%\begin{center}
%\includegraphics[width=0.75\textwidth]{Figures/KinDistr/KDvsM4lZoomed_Blinded_all_categories.pdf} %KD_Blinded_4l_Inclusive.pdf} %d_KD_4l_inclusive_.pdf}KD_Blinded_4l_Inclusive.pdf
%\includegraphics[width=0.455\textwidth]{Figures/KinDistr/KD_Blinded_4e_Inclusive.pdf} \\
%\includegraphics[width=0.455\textwidth]{Figures/KinDistr/KD_Blinded_4mu_Inclusive.pdf} 
%\includegraphics[width=0.455\textwidth]{Figures/KinDistr/KD_Blinded_2e2mu_Inclusive.pdf} \\
%\includegraphics[width=0.45\textwidth]{Figures/KinDistr/d_KD_M4L118130_4l_inclusive_.pdf}\\
%\includegraphics[width=0.45\textwidth]{Figures/KinDistr/d_2D_M4lVsKD_M4L100170_4l_.pdf}
%\includegraphics[width=0.45\textwidth]{Figures/KinDistr/d_2D_M4lVsKD_M4L1701000_4l_.pdf}
%\caption{
%One-dimensional distribution of the kinematic discriminant $\KD$ in the full mass range (blinded), for all final states (top left) and for 4e  (top right), 4mu (bottom left) and 2e2mu (bottom right) separately.  Points with error bars represent the data and stacked histograms represent expected distributions. The $125~\GeV$ Higgs boson signal and the $\cPZ\cPZ$ backgrounds are normalized to the SM expectation, the $\cPZ$+X background to the estimation from data.
%Top row: One-dimensional distribution of the kinematic discriminant $\KD$ in the full mass range (left) and in the mass region $118<\mllll<130~\GeV$ (right). Points with error bars represent the data and stacked histograms represent expected distributions. The $125~\GeV$ Higgs boson signal and the $\cPZ\cPZ$ backgrounds are normalized to the SM expectation, the $\cPZ$+X background to the estimation from data.
%Bottom row: 
%Distribution of the kinematic discriminant $\KD$ versus the four-lepton reconstructed mass $\mllll$ in the low-mass region 
%(left) and in the high-mass region (right). 
%The gray scale represents the expected relative density of $\cPZ\cPZ$ background plus Higgs boson signal for $\mH=125~\GeV$. The points show the data and the horizontal bars represent the measured mass uncertainties. 
%\label{fig:KDvsM4l2}}
%\end{center}
%\end{figure}
%=======


%=======
%\begin{figure}[!htb]
%	\vspace*{0.3cm}
%	\begin{center}
%		\includegraphics[width=0.45\textwidth]{Figures/KinDistr/D1jet_M4L118130_Unblinded_4l_InclusiveRun2.png}
%		\includegraphics[width=0.45\textwidth]{Figures/KinDistr/D2jet_M4L118130_Unblinded_4l_InclusiveRun2.png}\\
%		\includegraphics[width=0.45\textwidth]{Figures/KinDistr/DVH_M4L118130_Unblinded_4l_InclusiveRun2.png}
%		%\includegraphics[width=0.45\textwidth]{Figures/KinDistr/2018/DWH_M4L118130_Unblinded_4l_Inclusive.pdf} \\ %DWH_M4L118130_4l_inclusive_.pdf}
%		%\includegraphics[width=0.45\textwidth]{Figures/KinDistr/e_DZH_M4L118130_4l_inclusive_.pdf}
%		\caption{Distribution of the three production discriminants used for event categorization, in the mass region $118<\mllll<130~\GeV$. Points with error bars represent the data and stacked histograms represent expected distributions. The $125~\GeV$ Higgs boson signal and the $\cPZ\cPZ$ backgrounds are normalized to the SM expectation, the $\cPZ$+X background to the estimation from data. 
%			\label{fig:DprodC}}
%	\end{center}
%\end{figure}
%%=======
%
%%=======
%\begin{figure}[!htb]
%	\vspace*{0.3cm}
%	\begin{center}
%		\includegraphics[width=0.45\textwidth]{Figures/KinDistr/D1jetvsM4lZoomed_Unblinded_InclusiveRun2.png}
%		\includegraphics[width=0.45\textwidth]{Figures/KinDistr/D2jetvsM4lZoomed_Unblinded_InclusiveRun2.png}\\
%		\includegraphics[width=0.45\textwidth]{Figures/KinDistr/DVHvsM4lZoomed_Unblinded_InclusiveRun2.png}
%		%\includegraphics[width=0.45\textwidth]{Figures/KinDistr/e_2D_M4lVsDWH_M4L100170_4l_.pdf}
%		%\includegraphics[width=0.45\textwidth]{Figures/KinDistr/e_2D_M4lVsDZH_M4L100170_4l_.pdf}
%		\caption{Distribution of the three production discriminants used for event categorization versus the four-lepton reconstructed mass $\mllll$ in the low-mass region. The gray scale represents the expected relative density of $\cPZ\cPZ$ background plus Higgs boson signal for $\mH=125~\GeV$. The points show the data and the horizontal bars represent the measured mass uncertainties.
%			\label{fig:Dprod-corrC}}
%	\end{center}
%\end{figure}
%%=======
%
%
%\clearpage
%
%The Fig.\ref{fig:STXS_Categorization} shows number of expected and observed events in all Stage 1.1 sub-categories for full Run 2.     
%
%
%%=============                                                                                                                    
%\begin{figure}[!htb]
%	\vspace*{0.3cm}
%	\begin{center}	
%		\includegraphics[width=0.9\textwidth]{Figures/stxs/STXS_Categorization.pdf}
%		\caption{Distributions of the expected and observed number of events for all Stage 1.1 sub-categories described in Section~\ref{subsec:STXS_Categories} in the mass region $118<\mllll<130\GeV$ with Run 2 data. Points with error bars represent the data and stacked histograms represent the expected numbers of the signal and background events. The different SM Higgs boson signal production modes with $\mH=125\GeV$, denoted as ${\rm H}(125)$, and the $\cPZ\cPZ$ backgrounds are normalized to the SM expectation, the $\cPZ$+X background to the estimation from data. 
%			%The order in perturbation theory used for the normalization of the irreducible backgrounds is described in Section~\ref{sec:irrbkgd}.
%			\label{fig:STXS_Categorization}}
%	\end{center}
%\end{figure}
%%=======      

\clearpage 

%\subsection{Differential Cross Sections}
\subsection{Differential Cross Sections measurement results }
\label{sec:diff_xsec}
\input{Results/diff_xsec.tex}
\input{Results/eft_int_results.tex}
