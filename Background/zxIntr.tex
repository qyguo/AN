The reducible background for the $H\to ZZ\to 4\ell $ analysis, hereafter called $Z+X$, originates from processes which contain one or more nonprompt leptons in the four lepton final state. 
The main sources of nonprompt leptons are non-isolated electrons and muons coming from decays of heavy flavour mesons, misreconstructed jets (usually originating from light flavour quarks) and electrons from $\gamma$ conversions. 
In this discussion, we will consider a ``fake lepton'' any jet misreconstructed as a lepton and any lepton originating from a heavy meson decay.
Similarly, any electron originating from a photon conversion will be considered a ``fake electron''.

In the $H\to ZZ\to 4\ell $ analysis, the rate of these background processes is estimated by measuring the $f_{e}$ and $f_{\mu}$ probabilities for fake electrons and fake muons which do pass the {\bf loose} selection criteria (defined in Section~\ref{sec:eleReco} and~\ref{sec:muonReco}) to also pass the final selection criteria (defined in Section~\ref{sec:zzcandsel}).  
These probabilities, hereafter referred to as fake ratios or fake rates (FR), are applied in dedicated control samples in order to extract the expected background yield in the signal region. 

In the following section, two independent methods are presented to measure both the yields and shapes of the reducible background, called Opposite-Sign (OS) method and Same-Sign (SS) method. The final result combines the outcome of the two approaches. The methods are the same as in the 2016 analysis, although additional cross checks have been performed. 
