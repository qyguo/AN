One of the highest source of uncertainties in the analysis derives from the data-driven estimate of the reducible background.
There are three different sources to be taken into account:

\paragraph*{Statistical uncertainty}
The statistical uncertainty of the methods is driven by the limited size of the samples in the control regions where we measure ($Z + \ell$) and where we apply ($Z + \ell\ell$) the fake ratios and it is typically in the range of 1-10\%. 

\paragraph*{Systematic uncertainty due to fake rate variation}
A systematic uncertainty given by the variation of the expected yield considering up and down variations in the fake rate measurement has to be taken into account.

\paragraph*{Systematic uncertainty due to background composition}
The main source of the total uncertainty associated to the Z+X estimate is due to the different composition of the reducible background processes ($DY$, $t \bar{t}$, $WZ$, $Z\gamma^{(*)}$) in the regions where we measure and where we apply the fake ratios. On one hand, the OS method corrects for the resulting bias via the ``3P1F component" of its prediction. 
On the other had, the SS method corrects explicitely the electron fake rates by using the fraction of photon conversions, but no attempt is made to correct the muon fake rates.
%The closure tests presented here are used to assess a possible residual bias in the two methods.
To evaluate the sensitivity of the estimate to background composition, the residual bias in the two methods can be estimated by measuring the fake ratios for individual background processes in the $Z + \ell$ region in simulated samples: the weighted average of these individual fake ratios is the fake rate that we measure in simulation. The exact composition of the background processes in the 2P+2F region where we plan to apply the fake ratios can be determined from simulation, thereby the individual fake ratios can be reweighted according to the 2P+2F composition. The difference between the reweighed fake ratio and the average value can be used as an estimate of the uncertainty on the measurement of the fake ratios. 
The effect of this systematic uncertainty is propagated to the final estimates, and it amounts to about 32\% for $4e$, 33\% for $2e2\mu$ and 35\% for $4\mu$ final state. 

% \begin{table}[h]
% \scriptsize
%     \centering
%     \ction{
%     The fake ratios for individual background processes, the average fake ratio and the fake ratio reweighed according to the composition of backgrounds in 2P+2F region.}  
    
%     \begin{tabular}[!htb]{| l | c | c | c | c | c | c |} \hline
% Sample				& Light Jets		  & HF Jets& From $\gamma$ conv	    & average (in $Z+1\ell$) & reweighed (2P2F)	& uncertainty \\ \hline \hline
% electron fake ratio	& $0.012^{+0.001}_{-0.001}$ 	& $0.115^{+0.005}_{-0.006}$  & $0.293^{+0.043}_{-0.043}$  &$0.021^{+0.001}_{-0.001}$& $0.024^{+0.004}_{-0.004}$& 15\%    \\ 
% muon fake ratio 	& $0.057^{+0.003}_{-0.003}$ 	& $0.225^{+0.008}_{-0.008}$ & $0.003^{+0.455}_{-0.003}$ &$0.120^{+0.004}_{-0.004}$& $0.105^{+0.020}_{-0.017}$ & 13\% \\  \hline
% 	\end{tabular}
%     \label{tab:uncertFR}
% \end{table}

The final uncertainty associated to the combined Z+X yield is given by the sum in quadrature of these three contributions considering the full mass range of the $m_{4l}$ distribution in the SS method.
Table \ref{tab:zxUnc} shows the Z+X uncertainty for each final state in each year.

\begin{table}[!htb]
% \scriptsize                                                                                                                                                                                               
\begin{center}
    \begin{tabular}{ | l | c | c | c |  }        \hline
   Channel  & 4e       & 4$\mu$    & 2e2$\mu$ \\ \hline \hline
   2016     & 41$\%$   & 30$\%$    & 35$\%$   \\
   2017     & 38$\%$   & 30$\%$    & 33$\%$   \\
   2018     & 37$\%$   & 30$\%$    & 33$\%$   \\ \hline
    \end{tabular}
    \caption{ Systematic uncertainty associated to the Z+X estimate for each final state in all three years. }
    \label{tab:zxUnc}
\end{center}
\end{table}

A shape uncertainty is not associated to the Z+X estimate.
In order to evaluate the uncertainty on the $m_{4l}$ shape, we checked 
the difference between the shapes of predicted background distributions for all three channels and between the shapes given by SS and OS methods. 
%The envelope of the differences between these shapes of distributions is used as an estimate of the shape uncertainty. 
The uncertainty is estimated to be roughly in the range of 5\% - 15\%. 
Given that the difference of the shapes slowly varies with $m_{4l}$, it is taken as a constant versus $m_{4l}$ 
and is absorbed into the much larger uncertainty on the predicted yield of background events. 
%The shapes of predicted background $m_{4l}$ distributions for the three channels are shown in Figure~\ref{fig:SR_CombinedShapes} (left).	

